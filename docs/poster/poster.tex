\documentclass[landscape,a1paper,fontscale=0.33]{baposter}

\usepackage[portuguese]{babel}
\usepackage[utf8]{inputenc}
\usepackage{graphicx}
\usepackage{helvet}
\renewcommand{\familydefault}{\sfdefault}

% Necessário para que o BibTeX funcione com o Baposter
% Esse arquivo precisa estar na mesma pasta do poster.tex
% Baixe em:
% http://www.tex.ac.uk/tex-archive/biblio/bibtex/contrib/authordate/authordate1-4.sty
\usepackage{authordate1-4}

% The 'caption' package provides a nicer-looking replacement
\usepackage[labelfont=bf,textfont=it]{caption}

% Habilita subfigure.
\usepackage{subcaption}

% Habilita o uso de matrix.
\usepackage{amsmath}

% Habilita ambiente de algoritmos.
\usepackage[linesnumbered,portuguese]{algorithm2e}

% Divide igualmente o conteúdo da última página entre as duas colunas.
%\usepackage{flushend}

\usepackage{url}

% Copiado do template original:
\usepackage{relsize}
\usepackage{multirow}
\usepackage{booktabs}
\usepackage{multicol}
\usepackage{ae}

% Permite ajusta o espaçamento entre os itens de itemize
\usepackage{enumitem}

% Permite colocar direto o nome da imagem no includegraphics.
\graphicspath{{images/}}

% Distância entre colunas.
\setlength{\columnsep}{0.7em}

% Espessura da linha vertical que separa as colunas.
\setlength{\columnseprule}{0mm}

% Define the name of the game
\newcommand{\rawgamename}{Sweet Switches}
\newcommand{\gamename}{\textbf{\emph{\rawgamename}}}

%
% Challenges and Goals
% =====================
%
% Education
% ------
%
% - Computer literacy is more than just using computers, but also programming them
%     > similar metaphor to non-professional writers learning how to write
% - The skills involved in computer programming are useful for many other activitites
%     > thinking about thinking
%     > trial and error
% - Existing solutions are too focused on computer languages or too far from the
%   Brazilian culture
%     > brazilian state division and fauna
%
% Game
% ------
%
% - A puzzle in which players need to experimentaly design an ice cream factory to
%   produce the requested flavors
% - Simple mechanics based on path decisions
%     > with switchers and scales to conduct the products on the conveyors
% - Fantasy and aesthetics with a cartoonished style and funny animations
%
% Prototype
% ------
%
% - Experimentation of the basic mechanics and aesthetics, and interaction
%   methods
% - Presentation of the visual identity and main characters
%     > penguin (that comes from Argentina to the south region of Brazil) and
%       jaguar (that lives in the northest region of Brazil)
%
% Research
% ------
%
% - To be used for complementing the computer programming classes in the
%   Administration course from the University of São Paulo
% - To be used as a sample game in a PhD research interested in evaluating the
%   educational potential of games **** (to falando sobre a pesquisa do Bosco.
%   Talvez fosse legal perguntar pra ele - e pro professor! - sobre como escrever
%   isso aqui)
%
% Fun
% ------
%
% The game aims to produce a fun experience through different aspects:
% - Fantasy: an ice cream factory managed by the player with the help of penguin
%   assistants and other animals as truck drivers
% - Challenge: different configurations of conveyors in many levels, producing
%   initially simple problems but increasing complexity as the player progress in
%   the game
% - "Juiceness": pleasing designs and animations, with cartoonish and comical
%   styles strongly based on visual and auditory puns and incongruent elements
%   ("crazy machines")
% - Curiosity: elicitation of curiosity regarding the appearance and behavior of
%   new animals and devices (truck drivers and control devices unlocked as the
%   player progress in the game)
%
% Target
% ------
%
% - Children with at least 12 years old and adults
% - Playable in Windows, Linux, Android and iOS platforms
%     > and also via Web through Flash or HTML5
%
% Game Mechanics and Aesthetics
% =====================
%
% **** A ideia de construir um diagrama aqui com o "processo de jogo" é bastante
% bacana. Acho que aqui dá de colocar também algumas imagens do Danilo.
%
% Production
% =====================
%
% Engine
% ------
%
% - Currently using the HaxeFlixel engine
%     > open source engine based on the ActionScript 3 programming language
%       (Haxe + Flixel products)
%     > able to compile to all targeted platforms
%

\begin{document}
\begin{poster}{
    grid=false,
    eyecatcher=true, % Exibe imagem à esquerda do título.
    colspacing=0.7em,
    %%% Definindo as cores das caixas. %%%
    headerColorOne=black!45!cyan!15,
    borderColor=black!45!cyan!15,
    %%% Estilo da borda. %%%
    textborder=faded,
    %%% Estilo da sub-caixa de título de cada caixa. %%%
    headerborder=open,
    headershape=roundedright,
    headershade=plain,
    background=none,
    bgColorOne=cyan!10!white,
    boxheaderheight=2.35em,
    %%% Espaço reservado para o título, autores e imagens superiores. %%%
    headerheight=11em,
    %%% Número de colunas (de 1 a 6). %%%
    columns=5,
}
% Imagem superior esquerda.
{
    \includegraphics[height=10em]{sweetswitches}
}
% Título do pôster.
{
    \hfill
    \begin{minipage}{0.97\textwidth}
    In Sweet Switches players will face a series of puzzles inside ice cream
    factories. The objective is to attach devices to conveyor belts in order to
    respond to a production request. As those devices are analogous to
    programming statements, the game foster computational thinking.
    \end{minipage}
}
% Autores e nome da universidade.
{
    \scriptsize
    %\hfill%
    \centering
    \begin{minipage}{0.7\textwidth}
        \vskip 1.5em%
        %\centering
        \begin{minipage}{0.8\textwidth}
            Vinícius K. Daros, Luiz C. Vieira, Adalberto B. C. Pereira, Flávio S. C. da Silva\\[0.5em]
            Laboratory of Interactivity and Digital Entertainment Technology (LIDET)
            \\[0.15em]
            University of São Paulo - Brazil\\[0.5em]
            \texttt{\{vkdaros, lvieira, bosco, fcs\}@ime.usp.br}
        \end{minipage}
        \begin{minipage}{5em}
            \parbox[t][5em][t]{5em}{
                \begin{tabular}{l}
                    \includegraphics[width=5em]{usp-logo}\\
                    \includegraphics[width=5em]{universidade}
                \end{tabular}
            }
        \end{minipage}
    \end{minipage}

}
% Imagem superior direita.
{
%    \parbox[t][7.2em][b]{4.5em}{
%        \includegraphics[height=5.75em]{logo-ime}
%    }
}

\headerbox{Educational aspects}{name=education, column=0, row=0, span=1}{
    Existing approaches usually motivate learning with direct input of code.

    \vskip 1em%
    \gamename's goal is to promote the development of skills required for
    solving computational problems. Useful for tasks with:

    \begin{itemize}[noitemsep,topsep=0pt,parsep=0pt]
        \item Operation order
        \item Conditional execution
        \item Repetitions
    \end{itemize}

    \vskip 1em%
    The trial and error aspect of the game requires the players to:

    \begin{enumerate}[noitemsep,topsep=0pt,parsep=0pt]
        \item Follow the execution steps
        \item Understand where are the pieces causing undesired behavior
        \item Fix the problems. \textbf{This is essentially debugging}.
    \end{enumerate}
}

\headerbox{Tangential learning}{name=target, column=0, below=education, span=1}{
    Besides logical thinking, players may be able to learn about:

    \begin{description}[noitemsep,topsep=0pt,parsep=0pt]
        \item[Geography:] Find Brazilian cities on the map
        \item[Biology:] Discover animal species of various Brazilian regions
    \end{description}
}

\headerbox{Prototype}{name=prototype, column=4, row=0, span=1}{
    Intended for:

    \begin{itemize}[noitemsep,topsep=0pt,parsep=0pt]
        \item Playability evaluation
        \item Feedback on characters and visual identity 
        \item Participation in iGam4ER
    \end{itemize}

    \vskip 1em%
    Development tools used:

    \begin{description}[noitemsep,topsep=0pt,parsep=0pt]
        \item[Programing language:] Haxe
        \item[Framework:] Flixel
        \item[Levels:] Tiled
        \item[Art:] Gimp, Photoshop
    \end{description}
}

\headerbox{Targets}{name=target, column=4, below=prototype, span=1}{
    \begin{description}[noitemsep,topsep=0pt,parsep=0pt]
        \item[Audience:] $12+$ years old
        \item[Platform:] Web, Linux, Windows, Android, iOS
    \end{description}
}

\headerbox{Links}{name=links, column=4, below=target, span=1}{
    \begin{minipage}{\textwidth}
        \centering
        \includegraphics[width=0.7\textwidth]{onca}
    \end{minipage}

    \scriptsize
    \vskip 0.5em%
    \url{www.ime.usp.br/~lidet/sweet-switches}\\[0.5em]
    \url{www.github.com/vkdaros/igamer2014}

}

\headerbox{Construindo}{name=model, column=1, aligned=education, span=3}{
    \begin{multicols}{3}
        Durante uma volta de reconhecimento, o piloto virtual coleta dados
        através de sensores, como se fosse um robô. Ele tenta se manter no
        centro da pista, anda relativamente devagar (por volta de $60 Km/h$ em
        retas e $20 Km/h$ em curvas) e armazena a leitura dos sensores a cada 5
        metros. Ao fim da volta, os dados são processados. Assim, considerando
        pares consecutivos de medições, temos pequenos \textbf{segmentos} da
    \end{multicols}
}

\headerbox{Construindo2}{name=bot, column=1, above=bottom, span=2}{
    \begin{multicols}{3}
        Durante uma volta de reconhecimento, o piloto virtual coleta dados
        através de sensores, como se fosse um robô. Ele tenta se manter no
        centro da pista, anda relativamente devagar (por volta de $60 Km/h$ em
        retas e $20 Km/h$ em curvas) e armazena a leitura dos sensores a cada 5
        metros. Ao fim da volta, os dados são processados. Assim, considerando
        pares consecutivos de medições, temos pequenos \textbf{segmentos} da
    \end{multicols}
}

\end{poster}
\end{document}
