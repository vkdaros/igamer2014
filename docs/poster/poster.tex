\documentclass[landscape,a1paper,fontscale=0.33]{baposter}

\usepackage[portuguese]{babel}
\usepackage[utf8]{inputenc}
\usepackage{graphicx}
\usepackage{helvet}
\renewcommand{\familydefault}{\sfdefault}

% Necessário para que o BibTeX funcione com o Baposter
% Esse arquivo precisa estar na mesma pasta do poster.tex
% Baixe em:
% http://www.tex.ac.uk/tex-archive/biblio/bibtex/contrib/authordate/authordate1-4.sty
\usepackage{authordate1-4}

% The 'caption' package provides a nicer-looking replacement
\usepackage[labelfont=bf,textfont=it]{caption}

% Habilita subfigure.
\usepackage{subcaption}

% Habilita o uso de matrix.
\usepackage{amsmath}

% Habilita ambiente de algoritmos.
\usepackage[linesnumbered,portuguese]{algorithm2e}

% Divide igualmente o conteúdo da última página entre as duas colunas.
%\usepackage{flushend}

\usepackage{url}

% Copiado do template original:
\usepackage{relsize}
\usepackage{multirow}
\usepackage{booktabs}
\usepackage{multicol}
\usepackage{ae}

% Permite ajusta o espaçamento entre os itens de itemize
\usepackage{enumitem}

% Permite colocar direto o nome da imagem no includegraphics.
\graphicspath{{images/}}

% Distância entre colunas.
\setlength{\columnsep}{0.7em}

% Espessura da linha vertical que separa as colunas.
\setlength{\columnseprule}{0mm}

% Define the name of the game
\newcommand{\rawgamename}{Sweet Switches}
\newcommand{\gamename}{\textbf{\emph{\rawgamename}}}

\begin{document}

% Set the background to an image (poster_bg.pdf)
\background{
    \begin{tikzpicture}[remember picture,overlay]
    \draw (current page.north west)+(-2em,2em) node[anchor=north west]
    {\includegraphics[height=1.1\textheight]{images/poster_bg}};
    \end{tikzpicture}
}

\begin{poster}{
    grid=false,
    eyecatcher=true, % Exibe imagem à esquerda do título.
    colspacing=0.7em,
    %%% Definindo as cores das caixas. %%%
    headerColorOne=black!45!cyan!15,
    borderColor=black!45!cyan!15,
    %%% Estilo da borda. %%%
    textborder=none,
    boxshade=none,
    %%% Estilo da sub-caixa de título de cada caixa. %%%
    headerborder=open,
    headershape=roundedright,
    headershade=plain,
    background=user,
    bgColorOne=cyan!10!white,
    boxheaderheight=2.35em,
    %%% Espaço reservado para o título, autores e imagens superiores. %%%
    headerheight=11em,
    %%% Número de colunas (de 1 a 6). %%%
    columns=5,
}
% Imagem superior esquerda.
{
    \includegraphics[height=10em]{sweetswitches}
}
% Título do pôster.
{
    \hfill
    \begin{minipage}{0.97\textwidth}
    \vskip 1.5em%
    In Sweet Switches players will face a series of puzzles inside ice cream
    factories. The objective is to attach devices to conveyor belts in order to
    respond to a production request. As those devices are analogous to
    programming statements, the game fosters computational thinking.
    %TODO: logical bla
    \end{minipage}
}
% Autores e nome da universidade.
{
    \scriptsize
    %\hfill%
    \centering
    \begin{minipage}{0.94\textwidth}
        \vskip 1.1em%
        %\centering
        \begin{minipage}{0.515\textwidth}
            Vinícius K. Daros, Luiz C. Vieira, Adalberto B. C. Pereira, Flávio S. C. da Silva\\[0.5em]
            Laboratory of Interactivity and Digital Entertainment Technology (LIDET)
            \\[0.15em]
            University of São Paulo - Brazil\\[0.5em]
            \texttt{\{vkdaros, lvieira, bosco, fcs\}@ime.usp.br}
        \end{minipage}
        \begin{minipage}{0.1\textwidth}
            \textbf{Art:}\\
            Danilo Gabriel Rios and\\
            Salvador Oliva Junior
        \end{minipage}
        \begin{minipage}{0.27\textwidth}
            \parbox[t][6em][t]{0.44\textwidth}{
                \centering
                \begin{tabular}{c}
                    \includegraphics[width=6em]{usp-logo}\\
                    \includegraphics[width=6em]{universidade}
                \end{tabular}
            }
            \parbox[t][2.7em][b]{0.44\textwidth}{
                \includegraphics[height=6.5em]{LIDET-logo}
            }
        \end{minipage}
    \end{minipage}

}
% Imagem superior direita.
{
%    \parbox[t][7.2em][b]{4.5em}{
%        \includegraphics[height=5.75em]{logo-ime}
%    }
}

\headerbox{Educational aspects}{name=education, column=0, row=0, span=1}{
    Existing approaches to motivate learning of computer programming through
    entertainment are usually too focused on coding.

    \vskip 0.5em%
    \gamename's goal is to promote the development of skills required for
    solving computational problems. Useful for tasks with:

    \begin{itemize}[noitemsep,topsep=0pt,parsep=0pt]
        \item Operation order
        \item Conditional execution
        \item Repetitions
    \end{itemize}

    \vskip 0.5em%
    The trial and error aspect of the game requires the players to:

    \begin{enumerate}[noitemsep,topsep=0pt,parsep=0pt]
        \item Follow the execution steps
        \item Understand where are the pieces causing undesired behavior
        \item Fix the problems.
    \end{enumerate}
    \textbf{Similar to code debugging}
}

\headerbox{Tangential learning}{name=learning, column=0, below=education}{
    Besides logical thinking, players may be able to learn about:

    \vskip 0.75em%
    \begin{description}[noitemsep,topsep=0pt,parsep=0pt]
        \item[Geography:] \hfill\\
            Brazilian cities on the map
        \item[Biology:] \hfill\\
            Animals of the Brazilian fauna (penguins, jaguars, macaws, etc)
    \end{description}
}

\headerbox{Game elements}{name=elements, column=2, aligned=education, span=2}{
    \begin{minipage}{0.66\textwidth}
        \includegraphics[width=\textwidth]{GameElements}
    \end{minipage}
    \begin{minipage}{0.3\textwidth}
        Game elements are analogies for programming elements: conveyor segments
        work as commands, switches work as decisions, and scales work as are
        more complex decisions that allow repetitions (iterations).

        Starting and stopping the factory line are analogies for testing a
        program to verify if the sequence of commands and decisions produce the
        desired result.
    \end{minipage}
}

\headerbox{Flow of actions}{name=actions, column=2, below=elements, span=2}{
    \vskip 0.5em%
    \begin{minipage}{\textwidth}
        \centering
        \includegraphics[width=\textwidth]{GameActions}
    \end{minipage}
}

\headerbox{Motivation}{name=motivation, column=1, aligned=education}{
    A report prepared for the UK Computing Research Committee\footnote{Computing
    at School: the state of the nation, 2009} advocates that if on one hand
    learning how to use computers can be seen as similar to learning how to
    read, on the other hand learning how to program is similar to
    learning how to write. Both are skills that everyone should have, even
    though a minority will become professionals (writers or programmers).
}

\headerbox{Prototype}{name=prototype, column=4, aligned=education, span=1}{
    Open source project intended for:

    \begin{itemize}[noitemsep,topsep=0pt,parsep=0pt]
        \item Playability evaluation
        \item Feedback on characters and visual identity 
        \item Expose at \textit{iGam4ER 2014}
    \end{itemize}

    \vskip 0.75em%
    Development tools used:

    \begin{description}[noitemsep,topsep=0pt,parsep=0pt]
        \item[Programing language:] Haxe
        \item[Framework:] Flixel
        \item[Levels:] Tiled
        \item[Art:] Gimp, Photoshop
    \end{description}
}

\headerbox{Targets}{name=target, column=4, below=prototype, span=1}{
    \begin{description}[noitemsep,topsep=0pt,parsep=0pt]
        \item[Audience:] $12+$ years old
        \item[Platforms:] Web, Android, iOS, Linux, Windows
    \end{description}
}

\headerbox{Find more}{name=links, column=4, below=target, span=1}{
    \begin{minipage}{\textwidth}
        \centering
        \includegraphics[height=13em]{onca}
    \end{minipage}

    \scriptsize
    \vskip 0.5em%
    \url{www.ime.usp.br/~lidet/sweet-switches}
}

\headerbox{Screenshots}{name=screenshot, column=1, above=bottom, span=1}{
    \begin{minipage}{\textwidth}
        \centering
        \vskip 0.2em%
        \includegraphics[width=\textwidth]{screen1}

        \vskip 1em%
        \includegraphics[width=\textwidth]{screen2}
    \end{minipage}
}

\end{poster}
\end{document}
