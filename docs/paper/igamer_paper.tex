\documentclass{sigchi}

% Use this command to override the default ACM copyright statement (e.g. for
% preprints).
% Consult the conference website for the camera-ready copyright statement.

%% EXAMPLE BEGIN -- HOW TO OVERRIDE THE DEFAULT COPYRIGHT STRIP
%% (July 22, 2013 - Paul Baumann)
\toappear{LIDET is located inside the Department of Computer Science of the
          Institute of Mathematics and Statistics, at the University of S\~ao
          Paulo\\
          Rua do Mat\~ao, 1010, CEP 05508-090, S\~ao Paulo, SP, Brazil\\
          Telephone: +55 11 3091--9600\\

          Daros is a MSc student, Vieira and Pereira are PhD students, and
          Corr\^ea da Silva is their advisor.\\

          This manuscript was submitted in November 21st 2014 to the iGAM4ER
          2014 competition (http://igam4er.org/) taking place in Paris, France,
          in December 13th and 14th 2014.
}
%% EXAMPLE END -- HOW TO OVERRIDE THE DEFAULT COPYRIGHT STRIP
%% (July 22, 2013 - Paul Baumann)

% Arabic page numbers for submission.
% Remove this line to eliminate page numbers for the camera ready copy
% \pagenumbering{arabic}

% Load basic packages
\usepackage{balance}  % to better equalize the last page
\usepackage{graphics} % for EPS, load graphicx instead
\usepackage{times}    % comment if you want LaTeX's default font
\usepackage{url}      % llt: nicely formatted URLs
\usepackage{tabu}
\usepackage{subcaption}
\usepackage{enumitem}
\usepackage{multicol}

% llt: Define a global style for URLs, rather that the default one
\makeatletter
\def\url@leostyle{%
    \@ifundefined{selectfont}{\def\UrlFont{\sf}}{%
        \def\UrlFont{\small\bf\ttfamily}%
    }}
\makeatother
\urlstyle{leo}

% To make various LaTeX processors do the right thing with page size.
\def\pprw{8.5in}
\def\pprh{11in}
\special{papersize=\pprw,\pprh}
\setlength{\paperwidth}{\pprw}
\setlength{\paperheight}{\pprh}
\setlength{\pdfpagewidth}{\pprw}
\setlength{\pdfpageheight}{\pprh}

% Make sure hyperref comes last of your loaded packages,
% to give it a fighting chance of not being over-written,
% since its job is to redefine many LaTeX commands.
\usepackage[pdftex]{hyperref}
\hypersetup{
    bookmarksnumbered,
    pdfstartview={FitH},
    colorlinks,
    citecolor=black,
    filecolor=black,
    linkcolor=black,
    urlcolor=black,
    breaklinks=true,
}

% create a shortcut to typeset table headings
\newcommand\tabhead[1]{\small\textbf{#1}}

% path of images
\graphicspath{{./images/}}

% Shared affiliations
\def\sharedaffiliation{%
\end{tabular}
\begin{tabular}{c}}

% To keep the url in the same font of the institution address
\newcommand{\urlwofont}[1]{\urlstyle{same}\url{#1}}

% Column types for text
\newcolumntype{L}[1]{>{%
    \raggedright\let\newline\\\arraybackslash\hspace{0pt}}m{#1}%
}
\newcolumntype{C}[1]{>{%
    \centering\let\newline\\\arraybackslash\hspace{0pt}}m{#1}%
}
\newcolumntype{R}[1]{>{%
    \raggedleft\let\newline\\\arraybackslash\hspace{0pt}}m{#1}%
}

\newcommand{\refsectitle}[1]{\ref{#1}~(\nameref{#1})}

% End of preamble. Here it comes the document.
\begin{document}

% Define the name of the game
\newcommand{\gamename}{Sweet Switches}
\title{\gamename: a Digital Game to Foster Learning Computer Programming Skills}

\numberofauthors{4}
\author{
    \alignauthor Vin\'icius Kiwi Daros\\
        \email{vkdaros@ime.usp.br}
%
    \alignauthor Luiz Carlos Vieira\\
        \email{lvieira@ime.usp.br}
%
    \alignauthor Adalberto Bosco C. Pereira\\
        \email{bosco@ime.usp.br}
%
    \alignauthor Fl\'avio S. Corr\^ea da Silva\\
        \email{fcs@ime.usp.br}
%
    \sharedaffiliation
        \affaddr{Laboratory of Interactivity and Digital Entertainment%
                 Technology (LIDET)}\\
        \affaddr{\urlwofont{http://www.ime.usp.br/~lidet}}
}
\maketitle

%%%
 % TODO:
 % - Abstract
 % - Spell check
 % - Use "url" in all footnotes -> done!
 %%

\begin{abstract}
    TODO!
\end{abstract}

\keywords{
    games; education; computer programming; logical thinking; Brazilian fauna
}

\category{K.3.1.}{Computer Uses in Education}{Computer-assisted instruction%
                  (CAI)}
\category{K.8.0.}{Personal Computing}{Games}

\section{Introduction}
    \renewcommand{\gamename}{\textit{\textbf{Sweet Switches}}}
    \newcommand{\gamenamept}{\textit{\textbf{Sabores Seletos}}}

    In all developed countries, primary school children have at least one class
    a week when they use a computer \cite{Istrate2010}, but so far computers
    have been mostly employed to aid in other disciplines and as a productivity
    tool. So that it has been argued that young people should be educated not
    only in the application and use of technology, but also in
    \textit{how it works} and what are its \textit{fundamental principles}. A
    report prepared for the UK Computing Research Committee \cite{Jones2009}
    particularly mentions the fact that UK students are now becoming
    disenchanted with the computer aided learning because they arrive at school
    with a much richer background in ICT (Information and Communication
    Technology). The report also advocates that if on one hand learning how to
    use computers can be seen as similar to learning how to read, on the other
    hand learning how to program computers is similar to learning how to write.
    They are both skills that everyone should have, even though only a minority
    will become professionals (writers or computer programmers).

    Despite grounded arguments both in favor and against the use of computers by
    children \cite{Istrate2010,Setzer2001} and the relevance of the so called
    ``computer literacy'' or ``computational thinking'' trend
    \cite{Wing2006,Atwood2012}, the study of computer programming elements at
    primary and secondary school is increasingly being acknowledged as equally
    important as other disciplines such as math or science. The main reason is
    that the skills required for programming computers are definitely valuable
    for education, since they involve procedural thinking, problem solving
    through trial and error, creativity, thinking about thinking, and the
    analysis and exploration of data \cite{Kahn1999}.

    The experience the authors of this paper have had in teaching basic computer
    programming at the graduation course in Administration at the University of
    S\~ao Paulo corroborates that even when employing tools that the students
    are required to use in their daily routines and are accustomed with -- such
    as the Microsoft Office
    Excel\footnote{\url{http://office.microsoft.com/excel}} with the Visual
    Basic for Applications (VBA) programming language -- the task of learning
    how to use control structures in planning and solving problems by iterative
    execution is still perceived as a tedious activity even for older students.

    As consequence, the authors have been working on the development of a
    digital game called \gamename~(\gamenamept~in Portuguese), which has
    educational intentions mainly related to the development of skills necessary
    for programming computational tasks, \textit{but not directly involving
    coding}. The game mechanics is based on planning and building a production
    line that gets increasingly more complex as the player improves his or her
    understanding of how the control flow works. The next sections will describe
    existing similar playful environments and games that served as inspiration
    and details of the game under development.

\section{Similar Existing Products}
    % It would be nice to have some "glue text" about the section.
    \subsection{Playful Environments used for Teaching Programming}
        Playful environments employed to teach programming to children usually
        rely on motivating the creation of code from fun interactions with the
        real world or by allowing children to create their own content. One of
        the first examples of these environments is LabVIEW \cite{Erwin2000},
        used to help programing the behaviors of robots made of modular sensors
        and motors in the LEGO Mindstorms
        \footnote{\url{http://mindstorms.lego.com}}. It is a visual interface in
        which programs are built by stringing basic subroutines together in a
        block diagram, which can then be further reused as a new routines in
        higher levels.

        Another very popular example of playful environment is MIT's
        Scratch\footnote{\url{http://scratch.mit.edu/}} \cite{Resnick2009}. It
        is also a visual tool composed of graphical blocks that children snap
        together to move images or play sounds. The format of these blocks are
        carefully designed so they intuitively indicate syntax restrictions. For
        instance, a decision block (the ``if'' yellow block at
        figure~\ref{fig:scratch}) is C-shaped to suggest that other blocks can
        fit inside it.

    \subsection{Games used for Teaching Programming}
        There are also games used for teaching programming, but instead of
        allowing a free test-bed environment they are games that involve
        computer programming in the mechanics. One famous example is CodeCombat
        \cite{Saines2013}. This is an adventure game in which players control
        the hero character indirectly by programing behaviors through coding
        small functional actions such as move right, left, down and up, find
        nearest enemy and attack enemy. The game allows coding in many different
        programming languages, and includes important concepts like use of
        variables and repetition blocks.

        A very recent example is BBC's Doctor Who and and the Dalek
        \cite{BBC2014}. This is a platform game in which the player directly
        controls a Dalek cyborg
        \footnote{\url{http://en.wikipedia.org/wiki/Dalek}}, an extraterrestrial
        race from the British science fiction series \textit{Doctor Who}. The
        programming is used in the fantasy of the game as a manner of improving
        the cyborg abilities at the end of completed levels, like programming
        its jump or attack behaviors (as illustrated in
        figure~\ref{fig:dr-who}).

        \begin{figure}
            \centering
            \begin{subfigure}[!h]{0.5\columnwidth}
                \includegraphics[width=\textwidth]{scratch}
                \caption{Sample programs from Scratch}
                \label{fig:scratch}
            \end{subfigure}%

            \begin{subfigure}[!h]{0.7\columnwidth}
                \includegraphics[width=\textwidth]{dr-who}
                \caption{Repetition block from a program in Dr. Who and the
                         Dalek}
                \label{fig:dr-who}
            \end{subfigure}
            \caption{Programming in playful environments and games}
            \label{fig:samples}
        \end{figure}

    \subsection{Games involving Logistics}
        There are many games involving logistics, that is the management of the
        flow of resources between points to meet some requirements.

\section{The Construction of \gamename}
    Scratch's site audience is between the ages of 8 and 16 (peaking at 12),
    with also a sizable group of adult participants \cite{Resnick2009}.

    Without control flow structures such as ``IF'' commands in programming
    languages, it is simply not possible to construct a program that is useful
    because the resolution of problems depends on \textit{making decisions} in
    some scope.

    A problem with the existing approaches using entertainment to motivate
    learning programming skills is that they are too focused on direct input of
    code.
    \begin{quotation}
        \noindent
        \textit{The ``everyone should learn to code'' movement [...] assumes
                that coding is the goal. Software developers tend to be software
                addicts who think their job is to write code. But it's not.
                Their job is to solve problems.} \cite{Atwood2012}
    \end{quotation}

    The mechanics behind this puzzle is indeed very simple. And this is
    intentional. The game is doomed to become boring, as the player understands
    well enough on how to manage the flow of ice cream production and the game
    fantasy no longer brings novelty and curiosity.

\section{Educational Aspects Addressed}

\section{Concept Art}
    Here are some sketches of the concept art planned for \gamename.
    \begin{center}
        \includegraphics[width=0.9\columnwidth]{01}
        \includegraphics[width=0.9\columnwidth]{02}
        \includegraphics[width=0.9\columnwidth]{03}
        \includegraphics[width=0.9\columnwidth]{04}
        \includegraphics[width=0.9\columnwidth]{05}
        \includegraphics[width=0.9\columnwidth]{06}
    \end{center}

\section{Acknowledgments}
    The authors would like to thank CAPES (\textit{Coordena\c{c}\~ao de
    Aperfei\c{c}oamento de Pessoal de N\'ivel Superior}) and CRI (Center for
    Research and Interdisciplinarity) for the financial support, and the artists
    Danilo Gabriel Rios and Salvador Oliva Junior for the concept art.

% Balancing columns in a ref list is a bit of a pain because you
% either use a hack like flushend or balance, or manually insert
% a column break.  http://www.tex.ac.uk/cgi-bin/texfaq2html?label=balance
% multicols doesn't work because we're already in two-column mode,
% and flushend isn't awesome, so I choose balance.  See this
% for more info: http://cs.brown.edu/system/software/latex/doc/balance.pdf
%
% Note that in a perfect world balance wants to be in the first
% column of the last page.
%
% If balance doesn't work for you, you can remove that and
% hard-code a column break into the bbl file right before you
% submit:
%
% http://stackoverflow.com/questions/2149854/how-to-manually-equalize-columns-
% in-an-ieee-paper-if-using-bibtex
%
% Or, just remove \balance and give up on balancing the last page.
%
\balance

\bibliographystyle{acm-sigchi}
\bibliography{igamer_bib}
\end{document}
