\documentclass[a4paper]{scrartcl}

\usepackage[english]{babel}
\usepackage[utf8]{inputenc}
\usepackage{times}
\usepackage{graphicx}
\usepackage{url}
\usepackage[colorinlistoftodos]{todonotes}
\usepackage{multicol}
\usepackage{wrapfig}

% Look for images in ./images folder.
\graphicspath{{./images/}}

% Front page main information.
\title{\gamename}
\subtitle{Game Design Document}
\author{}
\date{\today}

\begin{document}

% Define the name of the game.
\newcommand{\gamename}{\emph{Ice Cream Factory}}

\maketitle

    \begin{quotation}
        \noindent
        \textit{The production of ice cream is a sweet business. There are so
                many flavors, so many toppings, so many combinations! The
                weather is hot and customers are avid for different types of ice
                cream. So hands-on and starting producing!}
    \end{quotation}

\section{Introduction}
    \gamename is a puzzle game for Windows, Linux, Web, Android and iOS in which
    the player needs to build an ice cream factory by correct placing cup
    dispensers, flavor and topping dosers, patch switchers and other devices on
    a predefined layout of conveyors, in order to produce the different types of
    ice creams in the requested quantities by customers.

    Once the production line is set-up by, the player can start the factory and
    watch the production happening. A truck driver character supports the
    fantasy and the mechanics by providing the player with the customer requests
    (posting orders in simple speech bubbles) and awaiting for the products,
    giving visual and auditory feedback on the success or failure of the puzzle.

    When the factory is running, empty cups are put in the conveyor belt by the
    cup dispenser and are carried by the conveyors in their indicated roll
    direction. When the cups get under other equipment they are manipulated by
    the intended action: for example, when under a flavor doser a scoop of ice
    cream is dispensed. Therefore, the player needs to put the devices on the
    conveyor belt \textit{in the correct order} so a specific type of ice cream
    is gradually composed. For instance, a chocolate with nuts ice cream can be
    produced by first conducting the empty cup under the chocolate flavor doser
    and then under the nuts topping doser, finally delivering it to the end of
    the conveyor belt to the truck driver.

    The puzzle difficulty increases with each level. Customers may require ice
    creams with more than one scoop or with extra toppings, which will require
    conducting the cups under a given device more than once. Also, the order
    posted by the truck driver may require different numbers of ice cream types
    (1 nuts-chocolate and 2 plain-chocolate, for instance). Additionally, the
    conveyors will form increasingly more complex shapes, requiring careful
    planing and experimentation for positioning the available equipment to
    properly satisfy the customer requests.

    The game targets both male and female players with at least 10 years old,
    without an upper limit of age.

    \subsection{Vision Statement}
        The player should be delivered with the following experiences:
        \begin{itemize}
            \item Satisfaction in solving challenging but lighthearted puzzles
            \item Amusement with the game visual art, music and sound effects
            \item Curiosity about the behavior of characters and the operation
                  of devices
        \end{itemize}

    \subsection{Objective}
        In each level the player has to set-up the available devices on the
        fixed conveyor belt layout in order to produce the amount of ice cream,
        in their different combinations, as requested by the truck driver.

    \subsection{Aesthetics}
        The game has a cartoony style both in its visual and auditory elements,
        with a 2D isometric projection. The graphical art is simple, colorful
        and beautiful and the sound effects and music are surprising, bouncy and
        funny. So, a complete production line feels and sounds like a ``crazy
        factory'' (while working, the factory will sound like vintage
        machinery\footnote{\url{http://www.soundsnap.com/node/36489}}).

    \subsection{Features}
        \begin{itemize}
            \item 3 base ice creams, 4 flavors, 5 toppings, 4 special toppings;
            \item 13 different devices;
            \item Over 100 levels;
            \item Mind blowing puzzles;
        \end{itemize}

    \subsection{Target platforms}
        \gamename is a simple and casual game, making it ideal for browser,
        tablets and PC. Players will interact with the game by touch (in case of
        tablets) or with the mouse cursor (in case of browser and PC). The
        targeted platforms are Android, iOS, Windows and Linux.

\section{Game Elements}

    \subsection{Characters}

        \subsubsection{The Player}
            The player has no avatar in the game, being intrinsically known as
            the owner of the ice cream factory. When interacting with the
            player, other characters will look towards the screen, for that
            matter. The player interacts with the game by touching the screen in
            version for mobile devices or by using the mouse cursor in the
            versions for the Web and other non-mobile operating systems.

        \subsubsection{Truck Drivers}
            Truck drivers are the supporting characters that enter each level
            driving an ice cream delivery truck, provide the ice cream order for
            the level and wait for the production to be concluded in order to
            delivery the ice cream to the customers. They are always in a hurry,
            because they need to delivery those ice creams yesterday! So while
            the player is thinking and placing the devices to build the factory,
            the truck drivers are doing silly idle animations to communicate the
            hurry impression: depending on the character, they might check their
            wristwatches, emit a sound effect that sounds like ``hurry
            up!''\footnote{\url{http://www.pond5.com/sound-effect/20644403/cartoon-game-voice-hurry.html}},
            tap their feet, etc.

            The truck drivers are represented by animals from the Brazilian
            fauna, dressed up in uniforms from the delivery company (with the
            same logo that is on the truck) dirty with ice cream. They do not
            speak in plain human language, but instead emit sounds that are
            barely similar to English or Portuguese. The ``hurry up!'' sound
            effect example from before is supposed to convey the sense of hurry
            more with intonation and anger than with direct language meaning.
            The idea is that the game may be easy to play for a broader audience
            without requiring specific localization.

            Besides the feedback on general things, the truck drivers
            communicate the orders of ice cream through visual symbols of the
            types of ice cream needed in a speech bubble and other behaviors
            through animations. During the running of the factory, when the
            truck drivers receive an ice cream that is in the order they will
            simply collect it and put it into the truck. If they receive an ice
            cream that is not in the order, they will discard it by tossing the
            ice cream to the ground and emitting a sound of complaint or
            annoyance (like a ``blah!''\footnote{\url{http://www.audiomicro.com/male-blah-human-vocal-male-royalty-free-stock-music-941876}}).
            When the truck is filled they will perform an inventory check to
            validate the puzzle solution given by the player. If the order is
            correct, they will emit a sound of satisfaction (like a
            ``wee!''\footnote{\url{http://www.pond5.com/sound-effect/30391857/cartoon-wee-voice.html}}
            or ``hooray!''\footnote{\url{http://www.soundsnap.com/node/49313}}),
            and hush away in their truck. Otherwise (if the order is not
            correct), they will get confused, scratch their heads and be
            startled by an animation of a robot arm that will take the truck and
            shake the ice creams out to a huge recycle bin. In that case, the
            level starts over and the players can try again without prejudice to
            their progress in the game.

            When a level is started, the order for ice creams in the speech
            balloon of the trucker driver is kept in the screen until the player
            starts positioning devices. However, if the trucker driver is
            touched (or clicked with the mouse cursor), the balloon is presented
            again for 5 seconds.

            New truck drivers will be unlocked as the player progress in the
            game. A screen section in the main menu will be provided for the
            player to consult the unlocked drivers, where she will be able to
            read information also from the real animals from the Brazilian fauna.

            The truck drivers conceived so far are:

            \begin{itemize}
                \item \textbf{Jo\~ao}\\
                      Based on the \textit{Rufous hornero}
                      bird\footnote{\url{http://en.wikipedia.org/wiki/Rufous_hornero}}
                      (\textit{Jo\~ao-de-barro} in Portuguese). He is depicted
                      wearing a tiny clay house as a hat (a comical reference to
                      the ``clay oven'' the bird uses for breeding and
                      sheltering) and has a huge wristwatch that is checked when
                      too long in idle.

                \item \textbf{Iara}\\
                      Based on the Macaw
                      bird\footnote{\url{http://en.wikipedia.org/wiki/Macaw}}
                      (\textit{Arara} in Portuguese). She is depicted with a
                      very neat uniform, and keeps doing her fingernails when
                      too long in idle -- alternating bored looks from the nails
                      and the player.

                \item \textbf{Paul\~ao}\\
                      Based on the Black lion tamarin\footnote{\url{http://en.wikipedia.org/wiki/Black_lion_tamarin}}
                      (\textit{Mico-le\~ao-preto} in Portuguese). He is depicted
                      as a very tiny monkey, with wide eyes and a attentive
                      face, always carrying a cup of coffee from which he
                      frantically sips when too long in idle. He also drives the
                      truck with half of his body outside the window, shaking
                      the cup of coffee in the free hand!

                \item \textbf{P\'ericles}\\
                      Based on the Jaguar\footnote{\url{http://en.wikipedia.org/wiki/Jaguar}}
                      (\textit{On\c{c}a-pintada} in Portuguese). He is depicted
                      as a very unprofessional driver, with the uniform shirt
                      outside the pants and some bumps and scratches in his
                      truck. He will eat all wrong ice creams instead of tossing
                      them away, and tap his feet in a fast pace when too long
                      in idle.
            \end{itemize}

            Other drivers will be designed later, in future versions of the game.

        \subsubsection{Factory Operators}
            Penguins work in the production line (since they can easily stand
            the cold -- they even enjoy it!), and thus keep walking around in
            the scene doing stuff: they carry clipboards, oil syringes and
            wrenches, and keep entering and leaving the scene, sometimes
            adjusting bolts in the positioned devices. They also clean up the
            ice creams that are tossed to the ground by the truck drivers and
            are wrong. Penguins are not original from the Brazilian fauna, but
            the \textit{Magellanic penguin}\footnote{\url{http://en.wikipedia.org/wiki/Magellanic_penguin}}
            arrives in the shore of some southern beaches coming from Patagonia
            (shared by Argentina and Chile). So in the fantasy of the game, they
            are temporary workers trying to make some money to enjoy the
            Carnival in Brazil.

            The factory operators are just used to improve the game fantasy,
            having a smaller impact in the interaction with the player. If they
            are touched by the player, they simply react by increasing the pace
            and running away from the player's finger (or mouse cursor).
            Sometimes, operators will be positioned nearby the delivery position
            in the scene when the level begins. They might then be run by the
            truck driver arriving with the delivery truck, as a comical pun. If
            that happens, the penguins hit are knocked out for 5 seconds (with
            little stars circling around their heads), to then get up and go
            back to work.

    \subsection{Conveyors}
        Each level has a fixed layout (a conveyor belt) made from straight and
        curved conveyors. Each conveyor is also drawn in a very cartoonish way,
        with a gadget made of boots used to make them move. The belts have
        arrows indicating their direction of movement, but the player can not
        directly interact with them to change their direction of movement or
        even stop them. However, the player can place devices in vacant
        positions along the conveyor belt.

    \subsection{Ice Cream Containers}
        The production line needs to be fed with items which will hold the ice
        cream. Without them our precious product would be spilled all over the
        place. To avoid that, usually the start point of the conveyor belt will
        have at least one device able to dispense on the line one of the
        following containers:

        \subsubsection{Ice cream cups}
            \begin{minipage}[t][3em][t]{\textwidth}
                \begin{wrapfigure}{l}{0.1\textwidth}
                    \vspace{-15pt}
                    \includegraphics[scale=1]{devices/pint}
                    \vspace{-20pt}
                \end{wrapfigure}

                Containers for regular ice cream. Depending on the level, the
                cups may be bigger in order to be capable of containing more
                scoops of ice cream obtained from multiple runs through dosers.
            \end{minipage}

        \subsubsection{Popsicle molds}
            \begin{minipage}[t][2em][t]{\textwidth}
                \begin{wrapfigure}{l}{0.1\textwidth}
                    \vspace{-15pt}
                    \includegraphics[scale=1]{devices/popsicle_molds}
                    \vspace{-25pt}
                \end{wrapfigure}

                Containers (with wooden sticks) for Popsicles. Differently than
                the ice cream cups, the Popsicle molds have a standard size.
            \end{minipage}

        \subsubsection{Banana plates}
            \begin{minipage}[t][2em][t]{\textwidth}
                \begin{wrapfigure}{l}{0.1\textwidth}
                    \vspace{-20pt}
                    \includegraphics[scale=1]{devices/banana_plate}
                    \vspace{-25pt}
                \end{wrapfigure}

                Special containers with a banana divided in half, used for
                preparing banana-splits.
            \end{minipage}

    \subsubsection{Production Devices}
        The production devices are responsible for bringing the production line
        to life! In other words, these are the devices that will dispense the
        containers, fill them with ice cream and cover them with toppings and
        candies.

        \subsubsection{Container Dispensers}
            These devices dispense containers for ice cream of a configured type
            and quantity. The player places one of these devices in the conveyor
            belt and configures it for which type of container is required.
            Action buttons are used to define the number of containers to be
            dispensed, ranging from 1 to 10. After the production line is
            started, the containers will be dispensed in regular intervals (each
            two conveyor cycles) until that number is reached.

        \subsubsection{Ice Cream Dosers}
            \begin{minipage}[t][8em][t]{\textwidth}
                \begin{wrapfigure}[5]{l}{0.16\textwidth}
                    \vspace{-20pt}
                    \includegraphics[scale=1]{devices/ice_cream_doser}
                    \vspace{-20pt}
                \end{wrapfigure}

                These devices dispense one single scoop of ice cream in the
                configured flavor into a container when one is located
                underneath it. They work with all types of containers in the
                same way, be it cups, Popsicle molds and banana plates. A doser
                can dispense ice cream in only one flavor, selected by the
                player before the production line is started. The flavor is
                selected by touching (or clicking) the device in a rotation
                manner (each touch selects the next flavor in a circular list).
                \\
                The existing ice cream flavors are: milk cream, strawberry,
                chocolate, mint and vanilla.
            \end{minipage}

        \subsubsection{Topping Dosers}
            \begin{minipage}[t][7em][t]{\textwidth}
                \begin{wrapfigure}[5]{l}{0.16\textwidth}
                    \vspace{-20pt}
                    \includegraphics[scale=1]{devices/topping_doser}
                    \vspace{-10pt}
                \end{wrapfigure}

                These devices work in the same way of the ice cream dosers, but
                dispensing topping flavors instead. They dispense only one
                measure of topping when a container is underneath it. The
                topping flavor dispensed is also selected by touching the
                device, before the production line is started.
                \\
                The existing topping flavors are: hot fudge (chocolate), hot
                fudge (caramel), chocolate chips, walnuts and sweet crumbs.
            \end{minipage}

        \subsubsection{Candy Dispensers}
            \begin{minipage}[t][6em][t]{\textwidth}
                \begin{wrapfigure}{l}{0.2\textwidth}
                    \vspace{-20pt}
                    \includegraphics[scale=1]{devices/special_topping_doser}
                    \vspace{-20pt}
                \end{wrapfigure}

                Similar to topping dosers, these devices dispense candies like
                marshmallows, Oreo crackers and waffle tubes. Only one candy is
                dispensed by each pass of a container underneath the device.
            \end{minipage}

    \subsection{Control Devices}
        The production devices have their limitations, mainly they are only able
        to dispense one scoop, measure or candy per time a cup pass underneath
        them. But the customers sometimes require two scoops of chocolate, an
        extra waffle tube or two measures of walnuts in their ice creams! Also,
        the number of devices available in each level is restricted. So, in
        order to cope with that, it is sometimes necessary to conduct the ice
        creams pots, Popsicle molds or banana plates under the production
        devices more than once. This is a big part of the challenges in
        \gamename.

        In order to control the path performed by the fixed layout of conveyors,
        the player uses control devices.

        \subsubsection{Switchers}
            \begin{minipage}[t][11em][t]{\textwidth}
                \begin{wrapfigure}{l}{0.2\textwidth}
                    \vspace{-20pt}
                    \includegraphics[scale=1]{devices/switcher}
                    \vspace{-15pt}
                \end{wrapfigure}

                Switchers are the simplest path control devices. They need to be
                placed in existing forks on the conveyor belt (where a track
                leads to two other possible tracks). They conduct a passing
                product to a given track and are toggled (changed) after the
                product passes, then conducting the next product to the other
                track. For instance, the first product passing is conducted to
                track A, when the switcher is then toggled so the second product
                passing is conducted to track B. The switcher is toggled again,
                so the third product passing will be conducted again to track A.
                And so on. The initial direction of the switcher (that is,
                before the production line is started), is defined by the player
                by touching the switcher action button.
            \end{minipage}

        \subsubsection{Weighing Scales}
            \begin{minipage}[t][10em][t]{\textwidth}
                \begin{wrapfigure}{l}{0.2\textwidth}
                    \vspace{-20pt}
                    \includegraphics[scale=1]{devices/scale}
                    \vspace{-20pt}
                \end{wrapfigure}

                The weighing scales are also switchers, but that change state
                (that is, the track to which they conduct products) based on the
                measured weighted of a product. Once the product arrives at a
                scale device, it is weighted. If the product's weight is greater
                than the device configured value, the product will be conducted
                to the ``heavier'' track and otherwise to the ``lighter'' track.
                \\
                For simplicity, all scoops of ice cream, measures of toppings
                and unities of candy weight one (1). The device configured
                weight is also set up by touching (or clicking) on up and down
                action buttons connected to it. The possible values are in
                integer numbers, ranging from 1 to 10.
            \end{minipage}

        \subsubsection{Counting Scales}
            \begin{minipage}[t][7em][t]{\textwidth}
                \begin{wrapfigure}{l}{0.2\textwidth}
                    \vspace{-20pt}
                    \includegraphics[scale=1]{devices/scale}
                    \vspace{-20pt}
                \end{wrapfigure}

                Similar to the weighing scales, but instead of weighting a
                product this type of switcher counts the number of products that
                have passed so far through it. Once the configured maximum
                number of products passed, it changes to the other path allowing
                one more product to pass. It then resets the counter back to
                zero. The maximum counter value is defined by the user by
                touching (or clicking) the action buttons, ranging from 1 to 10.
            \end{minipage}

        \subsubsection{Tasters}
            \begin{minipage}[t][5em][t]{\textwidth}
                \begin{wrapfigure}{l}{0.24\textwidth}
                    \vspace{-20pt}
                    \includegraphics[scale=1]{devices/taster}
                    \vspace{-20pt}
                \end{wrapfigure}

                Tasters are also switchers, but they check whether the ice cream
                flavor is the same of a specified one. If the flavor is the same
                as the one configured, the product goes to the ``yes'' track and
                otherwise to the ``no'' track. The pot is also conducted to the
                ``no'' track if it is empty. The player can set the taster
                verification flavor by also clicking the device, in a circular
                fashion.
            \end{minipage}

\section{Game Mechanics}

    \subsection{Interaction}
        The game is played by touch or mouse cursor. In each level, the game
        screen will present a fixed layout of conveyors forming the production
        line. Each straight segment will have an arrow indicating the direction
        in which the conveyors are running. In a separated area, the player will
        be presented with the devices and their quantities available for using
        in the building of her factory.

        The player selects or de-selects a device by touching or clicking it. A
        selected device is placed in game by touching or clicking again over a
        conveyor in the production line. Devices already in place can be moved
        by the same process: a selected device can be moved to another location
        by touching or clicking in the new location, or removed from the game
        back in the devices area by touching or clicking in there.

        Each device has one or two action buttons, allowing the player to set up
        their configurations by simply touching or clicking the buttons.
        Container dispensers, for instance, have one button that allows toggling
        among the existing options for ice cream containers. Counting and
        weighing switchers, for instance, have two buttons: one for increasing
        and another for decreasing the device configuration value. Regular
        switches also have a button, but for defining the default direction.
        Same for Tasters, which have a button for toggling among the flavors to
        be tested.

        The main screen has two other buttons: one for starting the production
        line, and another for reseting all the devices (put them back to the
        device area). Another button gives access back to the main menu. Once
        the production line is started, the conveyors will start to run, and all
        devices positioned will work when appropriated (container dispensers
        will work in a fixed rate, and dosers and control devices will work when
        an ice cream container is traveling through them).

    \subsubsection{Rewards}
        One first playing the game, only the first level is unlocked. A player
        has to succeed in solving the current puzzle to have access to the next
        level. Also, new truck drivers will be unlocked as the player progress
        in the game. Their profile sheets will be collected and displayed in a
        proper section, accessible from the game's main menu.

        The initial levels will be much easier to solve, but more difficult
        levels composed of more complex paths will allow different solutions,
        sometimes involving the use of less devices. In that case, the player
        will be granted with one to three starts depending on the solution
        achieved.

    \subsubsection{Example Levels}
        \includegraphics[width=\textwidth]{levels/legend}
        \includegraphics[width=\textwidth]{levels/01}
        \includegraphics[width=\textwidth]{levels/02}
        \includegraphics[width=\textwidth]{levels/03}
        \includegraphics[width=\textwidth]{levels/04}

\section{Concept art}
    Here there are a few images to show the visual identity planned for
    \gamename.

    \centering
    \begin{multicols}{2}
       \includegraphics[width=0.55\textwidth]{references/01}

       \hfill \includegraphics[width=0.43\textwidth]{references/02}
    \end{multicols}

    \begin{multicols}{2}
       \includegraphics[width=0.49\textwidth]{references/03}

       \includegraphics[width=0.49\textwidth]{references/04}
    \end{multicols}

    \begin{multicols}{2}
       \includegraphics[width=0.49\textwidth]{references/05}

       \includegraphics[width=0.49\textwidth]{references/06}
    \end{multicols}

\section{Similar games}
    In this section we can see some factory games which have the same conveyor
    concept present in \gamename.

    \begin{multicols}{2}
        \subsection{Cake Factory - Barbie Games}
            \includegraphics[width=0.49\textwidth]{similar_games/CakeFactory}

        \subsection{Toy Factory Fun}
            \includegraphics[width=0.49\textwidth]{similar_games/ToyFactoryFun}
    \end{multicols}

    \begin{multicols}{2}
        \subsection{Jewel Factory}
            \includegraphics[width=0.49\textwidth]{similar_games/JewelFactory}

        \subsection{Production Panic}
            \includegraphics[width=0.49\textwidth]{similar_games/ProductionPanic}
    \end{multicols}

    \begin{multicols}{2}
        \subsection{Candy Factory - LeeGT}
            \includegraphics[width=0.49\textwidth]{similar_games/CandyFactoryLeeGT}

        \subsection{Teddy Factory}
            \includegraphics[width=0.49\textwidth]{similar_games/TeddyFactory}
    \end{multicols}

    \begin{multicols}{2}
        \subsection{Robot Factory}
            \includegraphics[width=0.49\textwidth]{similar_games/RobotFactory}

        \subsection{Space Chem}
            \includegraphics[width=0.49\textwidth]{similar_games/SpaceChem}
    \end{multicols}

    \begin{multicols}{2}
        \subsection{Candy Conveyors}
            \includegraphics[width=0.49\textwidth]{similar_games/CandyConveyors}

        \subsection{The Cake Factory}
            \includegraphics[width=0.49\textwidth]{similar_games/TheCakeFactory}
    \end{multicols}

\section{Game name alternatives}
    \begin{itemize}
        \item Smart Factory
        \item Smart Cream
        \item Smart Ice Cream Factory
        \item Smart Split
        \item Ice Cream Factory
        \item Favorite Flavors (\textit{Sorvetes Sortidos})
    \end{itemize}

\end{document}
